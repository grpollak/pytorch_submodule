\begin{defnbox}\nospacing
    \begin{defn}[Autograd]\label{defn:autograd}\leavevmode\\
        \pythoninline{torch.autograd} is Pytorchs automatic differentiation module.
    \end{defn}
\end{defnbox}
\begin{defnbox}\nospacing
    \begin{defn}[\pythoninline{autograd.grad}]\leavevmode\\
       Calculates the gradient of an input tensor $x$ w.r.t.\ an output tensor $y$:
        \begin{mintlinebox}{python}
            x = torch.tensor(3.0, requires_grad=True)
            y = x ** 2
            grad_y_x = autograd.grad(y, x)
        \end{mintlinebox}
    \end{defn}
\end{defnbox}
\begin{defnbox}\nospacing
    \begin{defn}[\pythoninline{autograd.backward}]\leavevmode\\
        calculates all the gradients of all the tensors that are marked for gradient computation (\pythoninline{requires_grad=True})
        w.r.t.\ to a graph of a given tensor $y$ and accumulates the respective gradients into the \pythoninline{grad} fields of the tensors:
        \begin{mintlinebox}{python}
            x = torch.tensor(3.0, requires_grad=True)
            y = x ** 2
            # populate: x.grad and y.grad
            y.backward()
        \end{mintlinebox}
    \end{defn}
\end{defnbox}
\begin{attentionbox}{Attention}\nospacing \leavevmode
    \begin{itemizenosep}
        \item As of now, \pythoninline{autograd} only supports floating point and complex Tensor types.
        \item We can call \pythoninline{grad}/\pythoninline{backward} only once, unless with set \pythoninline{retain_graph=True}
        \item \pythoninline{y.backward()} is actually an alias for:\\ \pythoninline{autograd.backward(y, torch.ones_like(y))} as $\pdv{y}{y}=\vec{1}$.
    \end{itemizenosep}
\end{attentionbox}
\paragraph{Options}\label{para:options}
\begin{optionsbox}\nospacing
    \begin{option}[\pythoninline{retain_graph=None}]
          specifies whether to discard the graph after automatic differentiation.
    \end{option}
\end{optionsbox}


%%% Local Variables:
%%% mode: latex
%%% TeX-command-extra-options: "-shell-escape"
%%% TeX-master: "../../../../../../formulary"
%%% End:
