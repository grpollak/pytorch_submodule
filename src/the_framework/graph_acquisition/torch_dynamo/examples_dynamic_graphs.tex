\begin{examplebox}\nospacing
    \begin{example}[Data Dependent Control Flow]\label{example:data_dependent_control_flow}\leavevmode
        \begin{plaincodebox}{python}
            def fu(x):
                if x.sum() < 0:
                    return torch.tensor(0)
                return torch.tensor(1)
            x = torch.randn(5, 5)
            jit_fu = jit.trace(fu, (x,))
            fx_fu = fx.symbolic_trace.trace(fu, concrete_args={'x': x})
            dynamo_fu = dynamo.optimize(fu, (x,))
            jit_fu(-x) # != fu(x)
            fx_fu(-x) # != fu(x)
            dynamo_fu(-x) # == fu(x)
        \end{plaincodebox}
    \end{example}
\end{examplebox}
\begin{examplebox}\nospacing
    \begin{example}[Module Dependent Control Flow]\label{example:module_dependent_control_flow}\leavevmode\\
        After jit compilation\pythoninline{x = scipy.fft.dct(x.numpy())} and \pythoninline{x = torch.from_numpy(x)} will be treated as constants by \pythoninline{torch.fx} and \pythoninline{torch.jit}:
        \begin{plaincodebox}{python}
        def fu(x):
            x = x*2
            x = scipy.fft.dct(x.numpy())
            x = torch.from_numpy(x)
            x = x * 2
            return x
        x = torch.randn(5, 5)
        jit_fu = jit.trace(fu, (x,))
        fx_fu = fx.symbolic_trace.trace(fu, concrete_args={'x': x})
        dynamo_fu = dynamo.optimize(fu, (x,))
        jit_fu(-x) # != fu(x)
        fx_fu(-x) # != fu(x)
        dynamo_fu(-x) # == fu(x)
        \end{plaincodebox}
    \end{example}
\end{examplebox}
