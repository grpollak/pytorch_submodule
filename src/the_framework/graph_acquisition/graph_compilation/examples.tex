\begin{examplebox}\nospacing
    \begin{example}[TorchInductor\cref{defn:torchinductor}]\label{example:torchinductor}\leavevmode\\
        Using the options \pythoninline{trace.enabled} and \pythoninline{trace.graph_diagram} will create the folders:
        \begin{itemize}
            \item \texttt{torch\_compile\_debug/run\_\optc{<DATE\_TIME\_PID>}/aot\_torchinductor/}
            \item \texttt{model\_\_\optc{XX}\_\optc{(forward/backward)}\_\optc{XX}/output\_code.py}
        \end{itemize}
        that include the forward and backward code:
        \begin{plaincodebox}{python}
        import torch
        import torch._dynamo as dynamo
        from functorch.compile import make_boxed_func
        from torch.fx.passes.graph_drawer import FxGraphDrawer

        def fu(x: torch.Tensor):
            return torch.relu(x).neg() + torch.sin(x) ** 2

        dynamo.reset()
        cmp_f = torch.compile(fu,
            backend="inductor",
            options={"trace.enabled": True, "trace.graph_diagram": True}
        )
        device = "cpu" # device = 'gpu'
        x = torch.rand(1000, requires_grad=True).to(device)
        y = torch.ones_like(x)
        out = torch.nn.functional.mse_loss(cmp_f(x), y).backward()
        \end{plaincodebox}
    \end{example}
\end{examplebox}
