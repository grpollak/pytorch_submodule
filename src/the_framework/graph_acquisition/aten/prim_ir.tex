\begin{defnbox}\nospacing
    \begin{defn}[\hfill\hfill\exampleref{example:decomposing_aten_core_into_prims}\newline Prim IR\hfill\tcblack{[$\approx$250 ops]}]\label{defn:prim_ir}\leavevmode\\
        Prims IR is a lower level opset than core aten IR, and it further decomposes ops into explicit type promotion and broadcasting ops:
        Prims IR decomposes more complex operations such as \pythoninline{torch.ops.aten.mean} into a set of simpler operations
        such as \pythoninline{torch.ops.prims.mul.default}, \pythoninline{torch.ops.prims.sum.default} and \pythoninline{torch.ops.prims.div.default}.
    \end{defn}
\end{defnbox}
\begin{sectionbox}\nospacing
   \begin{proslist}
       \item decomposes Core Aten IR ops into a even smaller subset of operations making it even easier to write compilers for specific hardware.
   \end{proslist}
   \begin{conslist}
       \item Decomposing operators into lower and lower operations may lead to excess memory writes and function call overheads,
       which may lead to performance degradation.
   \end{conslist}
\end{sectionbox}
\begin{notebox}[Note]\nospacing
    The hope is that hardware compilers can take the smaller subset of operators and fuse them back up to hardware operations s.t. we don't
    suffer any degradation.
\end{notebox}
%%% Local Variables:
%%% mode: latex
%%% TeX-command-extra-options: "-shell-escape"
%%% TeX-master: "../../../../../formulary"
%%% End:
